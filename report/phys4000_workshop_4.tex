\documentclass{article}

% - style template
\usepackage{base}
\usepackage{siunitx}
\geometry{a4paper, margin = 0.75in}

% - title, author, etc.
\title{PHYS4000 - Workshop 4}
\author{Tom Ross - 1834 2884}
\date{\today}

% - headers
\pagestyle{fancy}
\fancyhf{}
\rhead{\theauthor}
\chead{}
\lhead{\thetitle}
\rfoot{\thepage}
\cfoot{}
\lfoot{}

% - document
\begin{document}

\tableofcontents

\listoffigures

\listoftables

\clearpage

\section{Potential Scattering}
\label{sec:potential-scattering}

Scattering calculations have been performed for a projectile, with charge
$z_{\rm{proj}}$, scattering off a structure-less potential (equivalently - a
one-state target) of the form
\begin{equation}
  \label{eq:potential}
  V\lr{r}
  =
  z_{\rm{proj}}
  \lr[\bigg]
  {
    1
    +
    \dfrac{1}{r}
  }
  \exp{- 2 r}
  .
\end{equation}
In these scattering calculations, the following parameters were constant:
$r_{\rm{max}} = 200$, $\rm{dr} = 0.001$ and $\ell_{\rm{min}} = 0$.
Two sets of calculations were performed:
\begin{enumerate}
\item
  With $\ell_{\rm{\max}} = 5$; for $z_{\rm{proj}} \in \lrset{-1, +1}$,
  for $E_{\rm{proj}} \in \lrset{E_{k} = \alpha + \beta k^{2}}_{k = 1}^{20}$
  with $\alpha, \beta$ such that $E_{1} = \SI{0.1}{\eV}$ and
  $E_{20} = \SI{50.0}{\eV}$, the calculation was performed, and the ICS and DCS
  curves extracted.

\item
  With $z_{\rm{proj}} = -1$, and $E_{\rm{proj}} = \SI{25.0}{\eV}$; for
  $\ell \in \lrset{0, \dotsc, 9}$, the calculation was performed, and the ICS
  and DCS curves extracted.
\end{enumerate}

\subsection{ICS Curves}
\label{sec:ics-curves}

The total and partial Integrated-Cross-Section (ICS) curves, extracted from the
first set of calculations, are shown for an electron and positron projectile in
\autoref{fig:ele-ics-curves} and \autoref{fig:pos-ics-curves} respectively.

\begin{figure}[h]
  \begin{center}
    \input{figure_ics_z--1_l-5.tex}
  \end{center}
  \caption[Electron ICS Curves]{
    The total ICS curve (shown in black) and the partial ICS curves (shown in
    blue-to-red) are presented, across projectile energies
    \SIrange{0.1}{50}{eV}, for an electron projectile, with
    $\ell_{\rm{min}} = 0$ and $\ell_{\rm{max}} = 5$.
    Note that the $y$-axis is presented in log-scale.
  }
  \label{fig:ele-ics-curves}
\end{figure}

\begin{figure}[h]
  \begin{center}
    \input{figure_ics_z-+1_l-5.tex}
  \end{center}
  \caption[Positron ICS Curves]{
    The total ICS curve (shown in black) and the partial ICS curves (shown in
    blue-to-red) are presented, across projectile energies
    \SIrange{0.1}{50}{\eV}, for a positron projectile, with
    $\ell_{\rm{min}} = 0$ and $\ell_{\rm{max}} = 5$.
    Note that the $y$-axis is presented in log-scale.
  }
  \label{fig:pos-ics-curves}
\end{figure}

\clearpage

\subsection{DCS Curves}
\label{sec:dcs-curves}

The Differential-Cross-Section (DCS) curves, extracted from the first set
of calculations, are shown for an electron and positron projectile in
\autoref{fig:ele-dcs-curves} and \autoref{fig:pos-dcs-curves} respectively.

\begin{figure}[h]
  \begin{center}
    \input{figure_dcs_z--1_l-5.tex}
  \end{center}
  \caption[Electron DCS Curves]{
    The DCS curves (shown in blue-to-red) are presented, across scattering
    angles \SIrange{0}{180}{\degree}, for an electron projectile, with
    projectile energies ranging across \SIrange{0.1}{50}{\eV}, and with
    $\ell_{\rm{min}} = 0$ and $\ell_{\rm{max}} = 5$.
    Note that the $y$-axis is presented in log-scale.
  }
  \label{fig:ele-dcs-curves}
\end{figure}

\begin{figure}[h]
  \begin{center}
    \input{figure_dcs_z-+1_l-5.tex}
  \end{center}
  \caption[Positron DCS Curves]{
    The DCS curves (shown in blue-to-red) are presented, across scattering
    angles \SIrange{0}{180}{\degree}, for a positron projectile, with
    projectile energies ranging across \SIrange{0.1}{50}{\eV}, and with
    $\ell_{\rm{min}} = 0$ and $\ell_{\rm{max}} = 5$.
    Note that the $y$-axis is presented in log-scale.
  }
  \label{fig:pos-dcs-curves}
\end{figure}

\clearpage

\subsection{DCS Curve Convergence}
\label{sec:dcs-curve-conv}

The Differential-Cross-Section (DCS) curves, extracted from the second set
of calculations, are shown in \autoref{fig:convergence-dcs-curves}.

\begin{figure}[h]
  \begin{center}
    \input{figure_dcs_convergence.tex}
  \end{center}
  \caption[Convergence of DCS Curves]{
    The DCS curves (shown in blue-to-red) are presented, across scattering
    angles \SIrange{0}{180}{\degree}, for an electron projectile, with
    projectile energy $E = \SI{25.0}{\eV}$, and $\ell_{\rm{min}} = 0$, with
    $\ell_{\rm{max}}$ ranging across \SIrange{0}{9}{}.
    Note that the $y$-axis is presented in log-scale.
  }
  \label{fig:convergence-dcs-curves}
\end{figure}

It can be seen that the DCS converges rather quickly for this projectile
energy of \SI{25.0}{\eV}.
A point of interest is that the DCS curve, for $\ell_{\rm{max}} = 0$, is
constant.
This is a consequence of the behaviour of the zeroth-order Legendre polynomials
$P_{\ell}\lr{\cos\theta}$, for which $P_{0}\lr{\cos\theta} = 1$.
To see this, note that the differential cross section, for this scattering
calculation, is of the form
\begin{equation*}
  \dv{\sigma}{\Omega}
  \lr{\theta}
  =
  \abs
  {
    f\lr{\vb{k}_{f}, \vb{k}_{i}}
  }^{2}
\end{equation*}
where $\vb{k}_{f}$ is such that $k_{f} = k_{i}$, and where
$\cos\theta = \hat{\vb{k}}_{f}\cdot\hat{\vb{k}}_{i}$, with the scattering
amplitude being of the form
\begin{equation*}
  f\lr{\vb{k}_{f}, \vb{k}_{i}}
  =
  -
  \dfrac{\pi}{k_{i}^{2}}
  \sum_{\ell = \ell_{\rm{min}}}^{\ell_{\rm{max}}}
  \lr{2\ell + 1}
  T_{\ell}\lr{k_{i}, k_{i}}
  P_{\ell}\lr{\cos\theta}
  .
\end{equation*}
Hence, where $\ell_{\rm{min}} = \ell_{\rm{max}} = 0$, we have that
\begin{equation*}
  f\lr{\vb{k}_{f}, \vb{k}_{i}}
  =
  -
  \dfrac{\pi}{k_{i}^{2}}
  \sum_{\ell = 0}^{0}
  T_{0}\lr{k_{i}, k_{i}}
  P_{0}\lr{\cos\theta}
  =
  -
  \dfrac{\pi}{k_{i}^{2}}
  T_{0}\lr{k_{i}, k_{i}}
\end{equation*}
whence
\begin{equation*}
  \dv{\sigma}{\Omega}
  \lr{\theta}
  =
  \dfrac{\pi^{2}}{k_{i}^{4}}
  \abs
  {
    T_{0}\lr{k_{i}, k_{i}}
  }^{2}
\end{equation*}
demonstrating the constant behaviour of the DCS curve for $\ell_{\rm{max}} = 0$.

\clearpage

\section{Derivation}
\label{sec:derivation}

\section{Dimensional Analysis}
\label{sec:dimensional-analysis}

\end{document}
